% This voodoo is needed for arXiv scripts and must appear within the first 4 lines
\pdfoutput=1
\documentclass[aps,prd,amsmath,floats,floatfix, twocolumn,
superscriptaddress,nofootinbib,showpacs,longbibliography]{revtex4-1}

% UTF8 always
\usepackage[T1]{fontenc}
\usepackage[utf8]{inputenc}
\usepackage{lmodern}

\usepackage{verbatim}

\usepackage[dvipsnames, usenames]{xcolor}
\definecolor{linkcolor}{rgb}{0.0,0.3,0.5}
\usepackage[hypertexnames=false, unicode, colorlinks=true, linkcolor=linkcolor,
citecolor=linkcolor, filecolor=linkcolor,urlcolor=linkcolor,
pdfusetitle]{hyperref}

%\usepackage[colorlinks, pdfborder={0 0 0}, plainpages=false]{hyperref}
\usepackage[all]{hypcap}
\usepackage{graphicx}
\usepackage{xspace}
%\usepackage[usenames,dvipsnames]{color}
\usepackage{amssymb}
\usepackage[normalem]{ulem} %for \sout
\usepackage{bm} % boldmath

% Better spacing
\usepackage{microtype}

\usepackage[english]{babel}
\usepackage{blindtext}

%
\graphicspath{%
  {figs/}%
  % More directories are added in braces, without commas between
}


\DeclareMathAlphabet{\mathpzc}{OT1}{pzc}{m}{it}

\newcommand{\roughly}{\mathchar"5218\relax\,} % Different from \sim in spacing
\newcommand{\into}{\!\times\!\relax} % Different from \times in spacing

% Macros for text changes
\newcommand{\red}{\textcolor{red}}
\newcommand{\vv}[1]{\textcolor{WildStrawberry}{VV: #1}}

\newcommand{\Note}[1]{\textcolor{blue}{\textbf{[#1]}}}
\newcommand{\h}{\mathpzc{h}}
\newcommand{\hlm}{\mathpzc{h}_{\ell m}}
\newcommand{\chieff}{\chi_{\mathrm{eff}}}
\newcommand{\chiPN}{\chi_{\mathrm{PN}}}


\newcommand{\bfemph}[1]{\emph{\textbf{#1}}}

\newcommand{\nn}{\nonumber}

\newcommand{\cd}{\nabla}
\newcommand{\pd}{\partial}
\newcommand{\lie}{\mathcal{L}}
\newcommand{\dd}{\mathrm{d}}

\newcommand{\TODO}[1]{\red{TODO: #1}}
\newcommand{\AddCite}{\red{[Needs citation]}}


% \newcommand{\mat}{{\tiny{\mathrm{mat}}}}
% \newcommand{\mat}{{(\mathrm{m})}}
\newcommand{\txt}[1]{{\textrm{\tiny{#1}}}}

%%%%%%%%%%%%%%%%%%%%%%%%%%%%%%%%%%%%%%%%%%%%%%%%%%%%%%%%%%%%%%%%%%%%%%%%%%%
\begin{document}

\title{The greatest science}

\newcommand{\AEI}{\affiliation{Max Planck Institute for Gravitational Physics
(Albert Einstein Institute), D-14476 Potsdam, Germany}}
\newcommand{\UMassD}{\affiliation{Department of Mathematics,
    Center for Scientific Computing and Data Science Research,
    University of Massachusetts, Dartmouth, MA 02747, USA}}
\newcommand{\Cornell}{\affiliation{Cornell Center for Astrophysics
    and Planetary Science, Cornell University, Ithaca, New York 14853, USA}}
\newcommand\CornellPhys{\affiliation{Department of Physics, Cornell
    University, Ithaca, New York 14853, USA}}
\newcommand\Caltech{\affiliation{TAPIR 350-17, California Institute of
    Technology, 1200 E California Boulevard, Pasadena, CA 91125, USA}}
\newcommand\Olemiss{\affiliation{Department of Physics and Astronomy,
    The University of Mississippi, University, MS 38677, USA}}

\author{Vijay Varma}
\email{vvarma@umassd.edu}
\UMassD

% Because hyperref only gets the *last* author, we need to be explicit.
\hypersetup{pdfauthor={Varma et al.}}

\date{\today}

%==========================================================================
\begin{abstract}
I scienced something, so now you get to read it.
\end{abstract}

\maketitle

%==========================================================================
\section{Introduction}
\label{sec:introduction}
I done did it. Some other people that done'd similar things:
\cite{Scott:2015rza}.

%==========================================================================
\section{Methods}
\label{sec:methods}
Here's how I done'd it.

%--------------------------------------------------------------------------
\begin{figure}[thb]
\includegraphics[width=0.43\textwidth]{string_cat.jpeg}
\caption{
What happened to Schrodinger's cat?
}
\label{fig:strings}
\end{figure}

%==========================================================================
\section{Results}
\label{sec:results}
Here's what I found after I done'd it.

And Maxwell said:
\begin{gather}
    dF = 0, \\
    d_{\star}F = \mu_0 \, J.
\label{eq:doppler_mass}
\end{gather}
and then there was light.


%==========================================================================
\section{Conclusion}
\label{sec:conclusion}
Here's what you should learn now that I done'd it.


%==========================================================================
\begin{acknowledgments}
% Randos
We thank Bob Loblaw~\cite{BobLoblaw} for useful discussions.
% VV
V.V.~acknowledges support from NSF Grant No. PHY-2309301, and the European
Union’s Horizon 2020 research and innovation program under the Marie
Skłodowska-Curie grant agreement No.~896869.
% LIGO
This material is based upon work supported by NSF's LIGO Laboratory which is a
major facility fully funded by the NSF.
% GWOSC
%This research made use of data, software and/or web tools obtained from the
%Gravitational Wave Open Science Center~\cite{GW_open_science_center}, a service
%of the LIGO Laboratory, the LIGO Scientific Collaboration and the Virgo
%Collaboration.
\end{acknowledgments}

%%%%%%%%%%%%%%%%%%%%%%%%%%%%%%%%%%%%%%%%%%%%%%%%%%%%%%%%%%%%%%%%%%%%%%%%%%%%%%%
\section*{References}
%%%%%%%%%%%%%%%%%%%%%%%%%%%%%%%%%%%%%%%%%%%%%%%%%%%%%%%%%%%%%%%%%%%%%%%%%%%%%%%
\bibliography{References}


\end{document}
